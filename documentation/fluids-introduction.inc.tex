% Introduction

This document describes the design process and usage of the \fluids\
application as result of assignment 6 of the course of Visualization~(2IV30) at
the Technical~University~of~Eindhoven.

\begin{quote}

In this assignment, you must construct a software tool for visualizing a time
dependent two dimensional vector field that represents the flow of a fluid.
This vector field is produced by a numerical simulation library called FFTW.
Along with the flow field, FFTW simulates the diffusion and advection of matter
in time. For an easy start, we provide a simple program that uses the FFTW to
simulate and visualize a fluid flow. Our visualization is limited to simple
hedgehogs colored by a blue-to-red colormap (see Course Material for details on
these two algorithms). The simulation of the flow is driven interactively by
the user, via mouse clicks in the window. The clicks inject local changes
(vortices, matter) in the flow.  The above software, written in C, can be
downloaded from:
\url{http://www.win.tue.nl/alext/COURSES/INFO_VIS/SOFTWARE/Smoke.zip}.

The assignment asks you to implement at least two of the following
visualization techniques: 

\begin{itemize} 

  \item Streamlines: show the flow by a number of instantaneous streamlines.
  You should decide yourself about where to start (and stop) the streamlines,
  how long the streamlines should be, how many to draw, how to draw them,
  etc. Some of these parameters may be exposed to the end user via a user
  interface. Others may be automatically set by the program itself. 


  \item Isolines: show the flow by displaying the isolines of the matter
  density. You should decide upon the number of isolines, distance between
  isolines, and way of drawing them, just as in the case of streamlines. 

  \item Height plots: show the flow by displaying a height plot (3D elevation
  graph) of the matter density. You should decide upon the viewing angle,
  height mapping, shading, and coloring of the 3D plot. 

\end{itemize}

For this assignment, students can work in groups of two persons. The
deliverable should contain: 

\begin{itemize}
  
  \item a report containing a description of the implemented methods, the
  choices made during the implementation, the problems found, limitations of
  the proposed solution, and a simple `user manual' of the implemented
  software, in total 5--10 pages. 

  \item the software: source code as well as running executable. 

\end{itemize}

Since the software we provide as a starting point is written in C under MS
Windows (with Visual C++), we recommend you to develop your solution in the
above setting. However, other implementations (Delphi, Java, Linux) are allowed
too, as long as all deliverables are present. The deadline of this assignment
is as for the other ones (before the spring term).

\end{quote}

It was chosen to implement streamlines and isolines, and to write the software
in C under Mac~OS~X and Linux. The final code compiles under Mac~OS~X and Linux
and should in theory also compile under Windows given the correct build options.

