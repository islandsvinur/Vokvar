% Problems

This section identifies the problems to be solved when implementing streamlines
and isolines in \fluids{}.

\subsection{Streamlines}

To draw a perfect streamline, one has to calculate an integral for an absent
function. In other words, this is impossible. The streamline can be
approximated however by use of numerical integration. 

There are various methods for doing this, Euler's and Runge-Kutta's techniques
being the most famous. While Euler's method is known to be not very accurate
when compared to Runge-Kutta's, the choice was put on the former.

\subsection{Isolines}

Isolines have the problem that it is impossible to determine the isovalues so
that the whole window is evenly filled with lines.

There are three ways of setting those values:

\begin{itemize}

\item By fixed value, simply a list of predefined ``clever'' values,
independent of the dataset.

\item By fixed number, taking the current maximal and minimal value of the
dataset, dividing it into a set number of isovalues.

\item By fixed points, taking a point in the dataset and determining the
isoline running through it.

\end{itemize}

All three methods are implemented using a simple version of the \textsc{conrec}
algorithm by Paul Bourke which was published in the Byte magazine issue of July
1987\footnote{\url{http://local.wasp.uwa.edu.au/~pbourke/papers/conrec/}}.

