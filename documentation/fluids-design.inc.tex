% Design

The \fluids\ component design is pretty straightforward. There are two main
components; the simulation and the visualization. 

\subsection{Simulation}

The basic functionality of this part is given by the \texttt{fluids-example.c}
file. A data structure called Simulation is created to hold the information
together. The listings in this section are taken from the file
\texttt{simulation.h}.

\lstinputlisting[linerange=12-58]{../simulation.h}

\subsection{Visualization}

\lstinputlisting[linerange=19-87]{../visualization.h}

\subsubsection{Smoke}

This was an existing visualization in the example and is integrated into
\fluids{}.

\lstinputlisting[linerange=6-6]{../visualization/smoke.h}

\subsubsection{Vectors}

This was an existing visualization in the example and is integrated into
\fluids{}.

\lstinputlisting[linerange=4-4]{../visualization/vectors.h}

\subsubsection{Streamlines}

\lstinputlisting[linerange=6-6]{../visualization/streamlines.h}

A streamline is generated by taking a point in the space and applying a
function to it to get the next point it will be after some set time step.

\fluids\ uses Euler's method, which means the function is of the form:

\begin{displaymath}
  \overline{x}_{t+\Delta{}} = \overline{x}_t + \overline{v}_{x_t} \times{} \Delta{}
\end{displaymath}

where $\overline{x}_t$ is the location of the particle to be traced at time
$t$, $\overline{v}_{\overline{x}_t}$ is the velocity of that particle at that
moment. $\Delta{}$ is the timestep size.

\subsubsection{Isolines}

The three methods of drawing imply three functions to call.

\lstinputlisting[linerange=6-9]{../visualization/isolines.h}

The CONREC algorithm is explained extensively in the paper by Paul Bourke. It
uses the observation that in a triangle, an isoline can only cut the triangle
in two if there is at least 1 point under the isoplane and at least one above
it.

Then by interpolation, a line is drawn over the triangle. Doing this with every
triangle of adjacent data points, the whole plane is filled and thus isolines
are constructed.
